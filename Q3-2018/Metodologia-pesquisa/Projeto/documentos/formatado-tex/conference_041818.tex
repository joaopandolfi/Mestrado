\documentclass[conference]{IEEEtran}
\IEEEoverridecommandlockouts
% The preceding line is only needed to identify funding in the first footnote. If that is unneeded, please comment it out.
\usepackage[utf8]{inputenc}
\usepackage{cite}
\usepackage{amsmath,amssymb,amsfonts}
\usepackage{algorithmic}
\usepackage{graphicx}
\usepackage{textcomp}
\usepackage{xcolor}
\usepackage{multirow}
\usepackage{array}
\usepackage{wrapfig}
\usepackage{multirow}
\usepackage{tabu}
\usepackage{longtable}

\def\BibTeX{{\rm B\kern-.05em{\sc i\kern-.025em b}\kern-.08em
    T\kern-.1667em\lower.7ex\hbox{E}\kern-.125emX}}
\begin{document}

\title{Metodologia de Pesquisa - Revisão sistemática\\
{\footnotesize Uma revisão sistemática sobre classificação de mosquitos utilizando aprendizado de máquina}
}

\author{\IEEEauthorblockN{1\textsuperscript{st} João Carlos Pandolfi Santana}
\IEEEauthorblockA{\textit{Mestrando em Ciência da computação (of UFABC.)} \\
\textit{Universidade Federal do ABC }\\
Santo André, SP \\
joao.pandolfi@ufabc.edu.br}
}

\maketitle

\begin{abstract}
\emph{background}
\emph{aims}
\emph{methods}
\emph{results}
\emph{conclusion}
\end{abstract}

\begin{IEEEkeywords}
Transference learning, machine learning, computational vision,  neural network, convolutional network, image classification cross validation ,classification, image segmentation, mosquito, mosquitos, insect, visão computacional, aprendizado de máquina, redes neurais, classificação de imagens, segmentação de imagem, insetos 
\end{IEEEkeywords}

\section{Introduction}
A identificação eficiente de mosquitos é importante para evitar vetorização de doenças. Alguns exemplos de doenças que são vetorizadas por mosquitos são: Dengue, Chikungunya e Febre amarela.
%
Uma forma de identificar o mosquito, é através de visão computacional, onde com uma imagem ou conjunto de imagens de entrada, seja possível identificar a espécie do inseto associando assim as doenças capazes de serem vetorizadas por ele.
%
Uma técnica de visão computacional largamente adotada, são as redes neurais, capazes de identificar características de baixo e alto nível em imagens. Neste trabalho, utilizaremos um modelo de rede mista com o conceito de \emph{transference learning}, no qual aproveita-se um conjunto de dados maior para treinamento de algum problema parecido e após um resultado suficientemente bom, adapta-se este aprendizado no problema chave, onde na maioria das vezes não se tem o volume de dados tão grande quanto o anterior.

Munidos destas técnicas, desenvolveremos um produto para identificação e classificação de mosquitos e seus possíveis vetores de doenças. Temos como objetivo migrar para uma plataforma mobile para facilitar o uso e acesso a ferramenta em regiões com baixo acesso a tecnologia de ponta. Também temos como objetivo contribuir com a instituição de pesquisa que nos cedeu os espécimes para coleta de imagens.

Para isso, esta revisão sistemática foi aplicada, visando encontrar trabalhos relacionados e materiais de apoio para o seguimento da pesquisa.

% ================================================================================= %

\section{Proposta} 
\subsection{Problema de pesquisa}
O problema abordado é a identificação por imagem de espécies de mosquito, de forma a agilizar e aumentar a eficiência de agentes de saúde e pesquisadores de campo em relação a epidemiologia e controle dos vetores de doença.

\subsection{Objetivos}
O objetivo principal é fazer uma revisão da literatura, onde possa reunir os trabalhos relacionados e partir de um problema em aberto ou otimizar alguma etapa ineficiente.
Outro objetivo é mapear as técnicas utilizadas e propor uma nova e mais eficiente para classificação de mosquitos. Tenho como objetivo final, desenvolver uma ferramenta mobile, no qual facilite a identificação do espécime capturado.
Por fim, também pretendo revisar conceitos de aprendizado de máquina, que possam ser aplicados na pesquisa, como por exemplo, \emph{transference learning} (transferência de aprendizado) onde possa utilizar sistemas especialistas em outras áreas, mas que possuam um certo grau de proximidade computacional do problema, de modo que sejam aplicados para otimização e melhoria no aprendizado do modelo.

\subsection{Hipótese}
A hipótese inicial é que não haja muita pesquisa relacionada ao tema proposto, devido ao fato dos estudos serem voltados, na sua maior parte, a doenças e migração dos vetores não na identificação computacional dos espécimes, também levo em conta que pelos insetos serem muito pequenos e dependerem de captura para análise, os pesquisadores que fazem a coleta já são capacitados para identificar o espécime, não abrindo tanta abertura para o problema de classificação destes. No entanto, viso desenvolver uma ferramenta para facilitar e reduzir o viés desta classificação, facilitando assim a análise em regiões que não possuam muitos especialistas e também permitir um pessoal menos treinado a fazer a primeira etapa da pesquisa, a identificação do espécime.

\subsection{Métodos}
Como a a hipótese se baseia em não ter muita pesquisa relacionada ao tema proposto,  procurarei conteúdo relacionado a classificação de imagem utilizando ferramental computacional que pretendo usar na pesquisa, assim como procurar material relacionado a estrutura corporal e classificação morfológica dos insetos a serem estudados.

\subsection{Resultados esperados}
Espero encontrar uma grande quantidade de conteúdo sobre classificação de imagens e conteúdo relacionado a morfologia de insetos. Assim como modelos e métodos previamente validados de \emph{transference learning} e visão computacional para basear minha pesquisa. 
Pretendo encontrar também, caso haja, pesquisas relacionadas ao tema proposto, de forma que possa seguir por caminhos consolidados e explorar conceitos ainda não estudados, assim, possa contribuir não só com o produto final produzido, mas também com técnicas computacionais novas ou aplicadas em áreas inexploradas.

% ================================================================================= %
\section{Métodos}

\subsection{Critério de seleção das fontes de busca}
O primeiro critério para seleção da fonte, foi a base ser internacional, assim eu tenho acesso a uma quantidade de artigos maior e consequentemente das maiores universidades e centros de pesquisa.

O segundo critério foram bases de dados relacionadas a publicações na área de ciência da computação, ou seja, não foram incluídas bases de dados que não tem a ver com o tema de pesquisa, esse critério foi adicionado para evitar ruído na pesquisa.

O terceiro e último critério, foi o uso do \emph{google scholar} para refinamento final da pesquisa, onde o objetivo foi pesquisar com um espectro mais amplo, vai de encontro com o segundo critério onde o objetivo é exatamente não ficar preso no espectro definido. O objetivo é somente verificar a relevância e qualidade da \emph{string} de busca, para garantir a assertividade, ou seja, se ao buscar numa base mais abrangente o resultado for próximo do encontrado na base mais específica, a \emph{string} está boa.

\subsection{Estratégia de busca}
A primeira etapa, foi a definição das strings de busca. O critério de qualidade de cada \emph{string}, foi definida pela quantidade de resultados em relação a correspondência do material apresentado com o tema da pesquisa. A sequência de \emph{strings} utilizadas se encontram na seção \ref{cap:string_busca}.

Após identificar as \emph{strings} mais promissoras, utilizei a ferramenta de busca avançada \emph{parsif.al}, onde adicionei os resultados em uma planilha no formato \emph{.csv}. Filtrei os resultados obtidos pelo título com os critérios de \emph{inclusão}~\ref{cap:crit_inclusao} e \emph{exclusão}~\ref{cap:crit_exclusao}, onde os resultados podem ser observados na Tabela~\ref{tab:artigos_selecionados}.

Por final, utilizei a ferramenta \emph{Google Scholar} para pesquisar artigos relacionados e verificar a qualidade das \emph{strings} de busca, ao ponto que dependendo da correlação dos resultados obtidos com o tema de pesquisa, pude verificar se as buscas anteriores foram assertivas.


\subsection{Strings de busca}
\label{cap:string_busca}
Strings de busca utilizadas durante a revisão sistemática

\begin{itemize}

    \item Parsifall
    \begin{enumerate}
        \item "transference learning" AND ("neural network" OR "neural networks") AND (mosquito OR mosquitos OR insect OR insects)
        
        \item ("transference learning" OR "machine learning" ) AND ("mosquito" OR "mosquito")
        
        \item ("classification" ) AND ("mosquito" OR "mosquito")
    
        \item \textbf{("neural network" ) AND ("classification") AND ("mosquito" OR "mosquitos")}
    
        \item \textbf{"transference" AND "learning" AND "machine"}
        

    \end{enumerate}

    \item Google Scholar
    \begin{enumerate}
        \item transference learning machine 
        \item transductive learning on insects classification
        \item image mosquito classification
        \item \emph{transductive learning}
    \end{enumerate}
    
\end{itemize}

\subsection{Critérios de inclusão}
\label{cap:crit_inclusao}
\begin{enumerate}
    \item Conter as palavras chave
    \item Ser relacionado com: 
    \begin{itemize}
        \item Machine Learning
        \item Transference Learning
        \item Transductive Learning
        \item Image Processing
        \item Image Classification
        \item Mosquito Classification
        \item Insect Classification
    \end{itemize}
\end{enumerate}

\subsection{Critérios de exclusão}
\label{cap:crit_exclusao}
\begin{enumerate}
    \item Fugir dos temas
    \item Não abranger sistemas de informação
    \item Não utilizar machine learning
    \item Não ter acesso ao pdf 
\end{enumerate}

\subsection{Estratégia para seleção}
Após o filtro pelo título dos trabalhos, fui para a leitura do \emph{abstract}, onde utilizei novamente os critérios de \emph{inclusão} e \emph{exclusão} para selecionar os artigos que fizessem sentido. Por fim, após as três etapas de filtragem citadas, parti para leitura dos arquivos selecionados
Pesquisei no google scholar para não ficar na bolha das bases selecionadas e usar o algoritmo de sugestão do google.

Adicionei uma \emph{label} indicando a importância 

\subsection{Avaliação da qualidade}
\begin{itemize}
    \item Quantidade de citações
    \item Correlação com o conteúdo pesquisado
    \item Profundidade da pesquisa
    \item Ferramentas utilizadas parecidas com as que pretendo usar
    \item Bons resultados
    \item Relevância da publicação
    \item Tipo de publicação
    \item Citação por artigo relevante, ou seja, um artigo relevante faz citação a este
\end{itemize}


\twocolumn
%\begin{longtable}{| m{3cm} | m{1cm} | m{1.2cm} | m{1cm} | m{1.2cm} |} 
\begin{longtable}{| m{5cm} | m{2cm} | m{4cm} | m{2cm} | m{2cm} |}

%\begin{table}[b]
\caption{Trabalhos selecionados}
%\begin{tabular}{| m{5cm} | m{2cm} | m{4cm} | m{2cm} | m{2cm} |} 
\hline
 \textbf{\textit{Título}} & \textbf{\textit{Autores}}& \textbf{\textit{Publicação}} & \textbf{\textit{Citações}}& \textbf{\textit{Comentário}} \\
\hline
Training of convolutional neural network using transfer learning for Aedes Aegypti larvae & Fuad M. & 1 August 2018 & 5 & Importância - alta \\ \hline 
A vision-based counting and recognition system for flying insects in intelligent agriculture & Zhong Y. & 9 May 2018 & 1 & Importância - média \\ \hline
Classification MFCC feature from culex and aedes aegypti mosquitoes noise using support vector machine & Lukman A. & 16 January 2018 & 1 & Material de apoio \\ \hline
Butterfly species recognition using artificial neural network & Alhady S. & 2018 & 0 & Importância - alta \\ \hline
Pre-trained convolutional neural networks as feature extractors toward improved malaria parasite detection in thin blood smear images & Rajaraman S. & 2018 & 1 & Importância - média \\ \hline
Automatic Identification of Malaria Using Image Processing and Artificial Neural Network & Kanojia M. & 2018 & 0 & Importância - média \\ \hline
Data driven prediction of dengue incidence in Thailand & Sumanasinghe N. & 2018 & 1 & Material de apoio \\ \hline
Artificial neural network as a classifier for the identification of hepatocellular carcinoma through prognosticgene signatures & Jujjavarapu S. & 2018 & 0 & Importância - média \\ \hline
Geometric morphometrics in mosquitoes: What has been measured? & Lorenz C. & October 2017 & 2 & Importância - Alta \\ \hline
Elman neural network trained by using artificial bee colony for the classification of learning style based on students preferences & Shuib N. & 1 September 2017 & 0 & Material de apoio \\ \hline
Learning to count mosquitoes for the sterile insect technique & Ovadia Y. & 13 August 2017 & 0 & Material de apoio \\ \hline
Automatic insect recognition using optical flight dynamics modeled by kernel adaptive ARMA network & Li K. & 16 June 2017 & 2 & Importância - Média \\ \hline
Sequential minimal optimization algorithm with support vector machine for mosquito larvae identification & Yusoff M. & May 2017 & 0 & Importância - Alta \\ \hline
A survey on image-based insect classification & Martineau M. & 1 May 2017 & 9 & Importância - Alta \\ \hline
Mosquito larva classification method based on convolutional neural networks & Sanchez-Ortiz A. & 3 April 2017 & 0 & Importância - Alta \\ \hline
Automated plasmodia recognition in microscopic images for diagnosis of malaria using convolutional neural networks & Krappe S. & 2017 & 0 & Importância - Média \\ \hline
Vision-based perception and classification of mosquitoes using support vector machine & Fuchida M. & 2017 & 5 & Importância -  Muito Alta \\ \hline
Identifying Aedes aegypti mosquitoes by sensors and one-class classifiers & Souza V. & 2017 & 0 & Importância - Alta \\ \hline
Artificial Neural Network applied as a methodology of mosquito species identification & Lorenz C. & December 01, 2015 & 12 & Importância - Muito Alta \\ \hline
Exploring Low Cost Laser Sensors to Identify Flying Insect Species: Evaluation of Machine Learning and Signal Processing Methods & Silva D. & 1 December 2015 & 14 & Importância - Alta \\ \hline
Flying Insect Classification with Inexpensive Sensors & Chen Y. & September 2014 & 29 & Importância - Alta \\ \hline
A hierarchical artificial neural system for genera classification and species identification in mosquitoes & Venkateswarlu C. & 2012 & 6 & Importância - Muito Alta \\ \hline
Classification and identification of mosquito species using artificial neural networks & Banerjee A. & December 2008 & 21 & Importância - Muito Alta \\ \hline
An aquatic insect imaging system to automate insect classification & Sarpola M. & 2008 & 16 & Importância - Alta \\ \hline
Classification MFCC feature from culex and aedes aegypti mosquitoes noise using support vector machine & Lukman A. & 16 January 2018 & 1 & Importância - Alta \\ \hline
Pre-trained convolutional neural networks as feature extractors toward improved malaria parasite detection in thin blood smear images & Rajaraman S. & 2018 & 1 & Importância - Média \\ \hline
A survey on image-based insect classification & Martineau M. & 1 May 2017 & 9 & Importância - Muito Alta \\ \hline
Vision-based perception and classification of mosquitoes using support vector machine & Fuchida M. & 2017 & 5 & Importância - Muito Alta \\ \hline
The image processing handbook: Seventh edition & Russ J. & 5 January 2016 & 5 & Material de apoio \\ \hline
Identification of medical plants using genetic algorithm and recurrent neural network & Malarkhodi S. & 2016 & 0 & Material de apoio \\ \hline
Microscopic image segmentation using hybrid technique for dengue prediction & Ghosh P. & 1 January 2016 & 0 & Material de apoio \\ \hline
Artificial Neural Network applied as a methodology of mosquito species identification & Lorenz C. & December 01, 2015 & 12 & Importância - Muito Alta \\ \hline
Mosquito vector monitoring system based on optical wingbeat classification & Ouyang T. & October 01, 2015 & 3 & Importância - Muito Alta \\ \hline
A tool for developing an automatic insect identification system based on wing outlines & Yang H. & 7 August 2015 & 11 & Importância - Alta \\ \hline
Flying Insect Classification with Inexpensive Sensors & Chen Y. & September 2014 & 29 & Material de apoio / Importância - Alta \\ \hline
On the role of pre and post-processing in environmental data mining & Gibert K. & 2008 & 16 & Material de apoio \\ \hline
Handbook of urban insects and arachnids & Robinson W. & 1 January 2005 & 36 & Material de apoio \\ \hline
Transductive learning for statistical machine translation & Nicola Ueffing National Research Council Canada Gatineau, QC, Canada nicola.ueffing@nrc.gc.ca Gholamreza Haffari and Anoop Sarkar Simon Fraser University Burnaby, BC, Canada &  &  & Material de apoio \\ \hline 


\hline
\multicolumn{4}{l}{$^{\mathrm{a}}$Sample of a Table footnote.}
%\end{tabular}
\label{tab:artigos_selecionados}
\end{longtable}
\twocolumn


% =========================================================================%

\section{Discussão}
Ao contrário do que previ na hipótese inicial encontrei bastante material relacionado a pesquisa, em contrapartida a maioria dos resultados das buscas foram sobre as doenças vetorizadas por mosquitos e estudos sobre migração e comportamento dos insetos. No quesito de aprendizado computacional, obtive bastante conteúdo de qualidade, o que irá ajudar bastante no desenvolvimento do projeto.

Tive bastante dificuldade ao selecionar conteúdos relacionados a computação, em vista que, muito material tem sido produzido na última década devido ao aumento do poder de processamento dos computadores e popularização das técnicas de aprendizado computacional. Para chegar aos artigos mais relevantes, dei ênfase aos mais citados e me vali de consultoria a professores da área para indicação e direcionamento da pesquisa.

Quanto aos trabalhos que têm como tema os insetos e sua morfologia, apesar de encontrar bastante artigo sobre o assunto, a maioria destes apontavam para livros, exceto os que tratavam de algum assunto específico estudado, como o \cite{Lorenz2017} que trata da geometria morfométrica de mosquitos (no qual, acredito eu, ser uma chave para identificação de \emph{features} da pesquisa da qual conduzirei). Os livros foram um pouco mais difíceis de se obter, em vista que são comprados, porém consegui acesso aos conteúdos e consegui adicionar a minha base de pesquisa.

% =========================================================================%

\section{Conclusão}
A pesquisa foi mais árdua do que imaginei que seria, contudo, obtive resultados que se mostram promissores, devido ao fato de alguns deles serem diretamente relacionados com a minha proposta de pesquisa e apresentarem resultados positivos. Assim como consegui materiais de suporte para desenvolvimento direcionado das etapas que precisarei transpor para concluir a pesquisa.

Em relação a classificação de mosquitos, encontrei estudos que desenvolveram o que propus estudar, porém não englobam todo o escopo que havia definido, os que tornam uma ótima base de pesquisa e pontapé inicial, assim como prova de conceito que de viabilidade. 

\subsection{Agradecimentos}
Agradeço aos professores da disciplina por me apoiarem e disponibilizarem material de apoio e tempo para que fosse possível o desenvolvimento deste trabalho.


% =========================================================================%


\section*{Acknowledgment}

The preferred spelling of the word ``acknowledgment'' in America is without 
an ``e'' after the ``g''. Avoid the stilted expression ``one of us (R. B. 
G.) thanks $\ldots$''. Instead, try ``R. B. G. thanks$\ldots$''. Put sponsor 
acknowledgments in the unnumbered footnote on the first page.

\section*{References}

Please number citations consecutively within brackets \cite{b1}. The 
sentence punctuation follows the bracket \cite{b2}. Refer simply to the reference 
number, as in \cite{b3}---do not use ``Ref. \cite{b3}'' or ``reference \cite{b3}'' except at 
the beginning of a sentence: ``Reference \cite{b3} was the first $\ldots$''

Number footnotes separately in superscripts. Place the actual footnote at 
the bottom of the column in which it was cited. Do not put footnotes in the 
abstract or reference list. Use letters for table footnotes.

Unless there are six authors or more give all authors' names; do not use 
``et al.''. Papers that have not been published, even if they have been 
submitted for publication, should be cited as ``unpublished'' \cite{b4}. Papers 
that have been accepted for publication should be cited as ``in press'' \cite{b5}. 
Capitalize only the first word in a paper title, except for proper nouns and 
element symbols.

For papers published in translation journals, please give the English 
citation first, followed by the original foreign-language citation \cite{b6}.

\begin{thebibliography}{00}
\bibitem{b1} G. Eason, B. Noble, and I. N. Sneddon, ``On certain integrals of Lipschitz-Hankel type involving products of Bessel functions,'' Phil. Trans. Roy. Soc. London, vol. A247, pp. 529--551, April 1955.
\bibitem{b2} J. Clerk Maxwell, A Treatise on Electricity and Magnetism, 3rd ed., vol. 2. Oxford: Clarendon, 1892, pp.68--73.
\bibitem{b3} I. S. Jacobs and C. P. Bean, ``Fine particles, thin films and exchange anisotropy,'' in Magnetism, vol. III, G. T. Rado and H. Suhl, Eds. New York: Academic, 1963, pp. 271--350.
\bibitem{b4} K. Elissa, ``Title of paper if known,'' unpublished.
\bibitem{b5} R. Nicole, ``Title of paper with only first word capitalized,'' J. Name Stand. Abbrev., in press.
\bibitem{b6} Y. Yorozu, M. Hirano, K. Oka, and Y. Tagawa, ``Electron spectroscopy studies on magneto-optical media and plastic substrate interface,'' IEEE Transl. J. Magn. Japan, vol. 2, pp. 740--741, August 1987 [Digests 9th Annual Conf. Magnetics Japan, p. 301, 1982].
\bibitem{b7} M. Young, The Technical Writer's Handbook. Mill Valley, CA: University Science, 1989.
\end{thebibliography}
\vspace{12pt}
\color{red}
IEEE conference templates contain guidance text for composing and formatting conference papers. Please ensure that all template text is removed from your conference paper prior to submission to the conference. Failure to remove the template text from your paper may result in your paper not being published.

\end{document}
